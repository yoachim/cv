%\documentclass[a4paper]{article}
\documentclass[11pt]{article}
% Seems like it does not support 9pt and less. Anyways I should stick to 10pt.
%\documentclass[a4paper, 9pt]{article}
%\topmargin-2.0cm

\usepackage{fancyhdr}
\usepackage{pagecounting}
\usepackage[dvips]{color}
% NEW COMMAND
% marginsize{left}{right}{top}{bottom}:
%\marginsize{3cm}{2cm}{1cm}{1cm}
%\marginsize{0.85in}{0.85in}{0.625in}{0.625in}
%\advance\oddsidemargin-0.65in
%\advance\evensidemargin-1.5cm
\textheight9.3in
%\textwidth6.75in

\usepackage[margin=1in]{geometry}

%\topmargin-2.0cm

%\textwidth6.5in
\newcommand\bb[1]{\mbox{\em #1}}
\def\baselinestretch{1.25}
%\pagestyle{empty}
\newcommand{\hsp}{\hspace*{\parindent}}
\definecolor{gray}{rgb}{0.3,0.3,0.3}

\linespread{1}

\begin{document}
\pagestyle{fancy}
%\pagestyle{plain}
% Leave Left and Right Header empty.
%\lhead{}
%\rhead{}
%\rhead{\thepage}
\lhead{\textcolor{gray}{\it Peter Yoachim}}
\rhead{\textcolor{gray}{\thepage/\totalpages{}}}
\renewcommand{\headrulewidth}{0pt}
\renewcommand{\footrulewidth}{0pt}
\fancyfoot[C]{\footnotesize \textcolor{gray}{http://staff.washington.edu/yoachim/}}




\begin{center}
{\Large \bf TEACHING STATEMENT}\\
\vspace*{0.1cm}
{\normalsize Peter Yoachim (yoachim@uw.edu)}
\end{center}
%\vspace*{0.2cm}



During my astronomy career, I have served as an instructor for a wide swath of age groups and skill levels as well as teaching large groups to one-on-one instruction.  
While a graduate student, I served as a TA for eight different undergraduate courses, including large and small introductory astronomy courses.  As a potdoc, I taught an Observational Techniques course for graduate students.
%I have also been a guest lecturer for upper division undergraduate courses.
I have given over 100 planetarium shows at the Holt Planetarium and University of Washington Planetarium.      
Finally, as a grad student and postdoc I have mentored several undergraduate research projects.
I believe successful teaching relies on three basic principles that I try to incorporate into all levels of my teaching.\\


%   Through these experiences, I have served as an instructor for a wide swath of age groups and skill levels as well as teaching large groups to one-on-one instruction.  



%\subsubsection*{Interaction}
\noindent{ \bf Interaction:}  Education research has shown that in an hour lecture, only around 10\% of the presented information is retained if students are only passive listeners.  No matter the venue or audience, I strive to make my teaching interactive.  The importance of interaction is most obvious in a planetarium where the dark comfortable room will lull an audience to sleep if there is no interaction.  

I strive to keep the level of interaction high while teaching, setting time aside for student discussion.  I also make a conscious effort to pause regularly for student questions and positively reinforce students who actively participate.  Once students see others getting rewarded for asking questions, they will be much more comfortable and eager to join the discussion. \\

%\subsubsection*{Collaboration}

\noindent{\bf Collaboration:} Humans have a natural tendency to fall into ``them vs. us'' or ``in-group vs. out-group'' thinking.  As an instructor, I might not have much in common with my students, and thus risk falling into the trap of ``them vs. us.''  Teachers often blame students as lazy; students in turn blame teachers as unreasonable.  

Many simple things can be done to build a sense of collaboration.  I like to schedule office hours when students will be working on homework.  I try to be available via email in off-hours.  I usually set flexible homework policies--emphasizing that I'm not concerned with when/how they learn the material, I simply want to work with them so they do learn.  If a student questions the way I graded an exam, I like to give a personalized written explanation.  When students recognize I am a resource that is willing to help them learn, they often reciprocate by being willing to work harder in class. \\

%\subsubsection*{Passion}

\noindent{\bf Passion:}   Perhaps the most important role I play as an instructor is conveying the excitement that comes from studying the universe on the largest scales.  When teaching, I regularly emphasize the broader context and historical motivation for the topic at hand.  I also work to incorporate my own research and other active research programs into courses.  Students will often ask ``Why should I care?''  I feel it is important to answer this question honestly and often while teaching.   \\

%I have to relate a research project to the larger context--otherwise a student will become despondent hunting through software bugs.  

These are the lessons I've learned from my years as a teacher.  It is not enough to lecture, students must interact with you and each other to effectively learn.  It is not enough to assign and grade--a teacher must be an active collaborator in the learning process.  Finally, it takes a motivated teacher to breathe life into a subject and prevent it from becoming drudgery.  A slap-dash lecture will not generate student interaction, and a poorly planned research project will be boring.  I have not found many shortcuts--being interactive, collaborative, and passionate takes effort, but that effort is rewarded with students who become more engaged with the material.


%I have attached several student evaluation forms I have received to illustrate how my students have positively responded to my teaching style.


%\vspace{0.5cm}



\end{document}

Many students utilize astronomy courses to fulfill science breadth requirements, and these courses may be their only exposure to the physical sciences.  Therefore it is vital to impart not only the basics of astronomy, but also the general principles of science in a short amount of time.  

%As a TA at UW, I tried to incorporate teaching methods that have been empirically shown to improve student understanding.  

Research in psychology suggests that learning is a constructive process where students relate new knowledge with concepts that are already known.  As a TA at UW, I incorporated the principles of active learning by encouraging student participation rather than relying exclusively on lecturing.  One example of the active learning techniques I employ is peer instruction--I am constantly impressed at how students who have just learned a new fact can gage the level of understanding in their peers and explain the material accordingly.  

There are some common difficulties encountered when trying to implement active learning in an astronomy class. Many students believe that mathematical ability is an innate skill rather than a learned ability, which leads them to be very hesitant to ask questions in class.  They fear their questions could reveal an innate ignorance to their peers.  I strive to create a classroom environment where student questions and interaction are rewarded.  Giving planetarium shows to grade school students, I was amazed at their enthusiasm and ability to ask pertinent questions nearly non-stop.  When working with undergraduates, I try to revive that sense of passion and curiosity in my students and get them back to actively asking questions.  Another common difficulty is motivation--many students take astronomy purely to fulfill university graduation requirements.  This can create an adversarial attitude where students see instructors as standing in the way of getting their degree.  As an instructor, I try to foster a collaborative environment where I am a resource to help the learning process.

In addition to these active learning techniques, I attempt to integrate everyday topics into my introductory astronomy courses.  For example, the difficulties in making an accurate political poll or census survey are incredibly similar to the difficulties in making an accurate astronomical survey.  Most current polls do not include those who only have a cell phone.  Similarly, astronomical surveys are always biased to missing very faint objects.  Relating these familiar statistical and mathematical problems to the more abstract problems of astronomy has been a useful strategy in facilitating student learning.  


Besides teaching introductory courses, I would like an opportunity to teach a seminar course on current astronomical research.  A similar course sparked my interest in astronomy, and I believe it is an easy way to engage students who are apprehensive about physical science while also providing an opportunity to encourage top students to pursue science majors.  %XXX-more planetarium stuff.

%I would also like to help develop an astronomy laboratory course for advanced undergraduates.  A quality laboratory course is crucial for preparing students to participate in research opportunities like NSF funded REUs.  A natural first lab course would include CCD observations and analysis. Other possibilities include labs on astronomical simulations or data mining techniques on archival observations.

I have served as an instructor in a wide variety of situations.  With the Oakland Community Networking Project, I taught employees in non-profit organizations how they could utilize internet tools in their work.  I have given hundreds of planetarium shows to groups of all ages.  At the University of Washington, I taught several undergraduate courses, ranging from large lecture courses to small discussion sections, and also worked with undergraduates one-on-one as they completed research projects.  I look forward to further refining my teaching techniques and working with new students.












%Many of the popular teaching methods like active learning and peer instruction are effective, but tend to show the largest gains in average or struggling students.  I believe the classroom should also be used to challenge (and recruit) advanced students as well.  Physics majors have the highest acceptance rate at medical schools, and this can be used to entice top students.  I would also advocate offering an astronomy seminar course.  A seminar can create a more informal environment, and can bring in a wider range of students who might never consider taking a Physics course.  



%I believe in implementing teaching strategies that have been shown to work for the majority of students, while keeping an eye out for exceptional students who can be encouraged to take ever more challenging courses and are prime examples of those who should major and minor in astronomy or Physics.





% LocalWords:  yoachim internet
